\documentclass[11pt]{article}

\usepackage{fullpage}
\usepackage{amsmath}
\usepackage{amssymb}
\usepackage{amsthm}
\usepackage{fancyhdr}
\usepackage{algorithm}
\usepackage{algorithmic}
\usepackage{bm}
\usepackage[usenames,dvipsnames]{xcolor}
\usepackage{hyperref}
\usepackage{extarrows}
\usepackage{fontspec}
\setmainfont{Times New Roman}

\newtheorem{theorem}{Theorem}[section]
\newtheorem{lemma}[theorem]{Lemma}
\newtheorem{corollary}[theorem]{Corollary}
\newtheorem{proposition}[theorem]{Proposition}
\newtheorem{definition}[theorem]{Definition}
\newtheorem{conjecture}[theorem]{Conjecture}
\newtheorem{remark}[subsection]{Remark}

%%
\newcommand\numberthis{\addtocounter{equation}{1}\tag{\theequation}}

%% define new symbols
\def\bx{\bm{x}}
\def\bb{\bm{b}}
\def\ba{\bm{a}}
\def\bc{\bm{c}}
\def\bf{\bm{f}}
\def\by{\bm{y}}
\def\bu{\bm{u}}
\def\bv{\bm{v}}
\def\BW{\bm{W}}
\def\BA{\bm{A}}
\def\bz{\bm{z}}
\def\BZ{\bm{Z}}
\def\BH{\bm{H}}
\def\BL{\bm{L}}
\def\BU{\bm{U}}
\def\BV{\bm{V}}
\def\BB{\bm{B}}
\def\BC{\bm{C}}
\def\BD{\bm{D}}
\def\BE{\bm{E}}
\def\BW{\bm{W}}
\def\BQ{\bm{Q}}
\def\BG{\bm{G}}
\def\BA{\bm{A}}
\def\BX{\bm{X}}
\def\BY{\bm{Y}}
\def\BQ{\bm{Q}}
\def\BI{\bm{I}}
\def\BR{\bm{R}}

%% define new brackets
\def\la{\left\langle}
\def\ra{\right\rangle}
\def\ln{\left\|}
\def\rn{\right\|}
\def\lb{\left(}
\def\rb{\right)}
\def\lsb{\left[}
\def\rsb{\right]}
\def\lcb{\left\{}
\def\rcb{\right\}}

%%
\DeclareMathOperator*{\argmin}{arg\,min}
\DeclareMathOperator*{\argmax}{arg\,max}

%%
\title{Homework I}
\author{Jialun Shen\\
  16307110030 \\ }
%\date{}


\begin{document}
\maketitle
%------------------------------------
%\begin{abstract}
%\end{abstract}
%-------------------------------------
%=====================
\section{Proof for Properties of Common Functions}

The \textbf{logistic sigmoid} funcition is
\begin{equation}
\sigma(x) = \frac{1}{1+\exp(-x)}. \tag{3.30}\label{sigmoid}
\end{equation}

The \textbf{softplus} function is
\begin{equation}
\zeta(x) = \log(1+\exp(x)). \tag{3.31}\label{softplus}
\end{equation}

Multiply both the denominator and numerator of the fraction on the right side of \autoref{sigmoid} with $\exp(x)$, we have
\begin{equation}
\sigma(x) = \frac{\exp(x)}{\exp(x)+1} = \frac{\exp(x)}{\exp(x)+\exp(0)}. \tag{3.33}
\end{equation}

The derivative of $\sigma(x)$ is
\begin{align}
\frac{d}{dx}\sigma(x) &= \frac{d}{dx}\left(1+e^{-x}\right)^{-1} \notag\\
&= \left(1+e^{-x}\right)^{-2}e^{-x} \notag\\
&= \frac{1}{1+e^{-x}}\cdot\frac{e^{-x}}{1+e^{-x}} \notag\\
&= \sigma(x)(1-\sigma(x)). \tag{3.34}
\end{align}

Also, we have
\begin{align}
1 - \sigma(x) = 1 - \frac{1}{1+\exp(-x)} = \frac{\exp(-x)}{1+\exp(-x)}
= \frac{1}{\exp(x)+1} = \sigma(-x) \tag{3.35}\\
\log\sigma(x) = \log\frac{1}{1+\exp(-x)} =  -\log(1+\exp(-x))
= -\zeta(-x). \tag{3.36}
\end{align}

The derivative of $\zeta(x)$ is
\begin{align}
\frac{d}{dx}\zeta(x) = \frac{d}{dx}\log(1+\exp(x)) 
= \frac{e^{x}}{1+e^x} = \frac{1}{1+e^{-x}} = \sigma(x). \tag{3.37}
\end{align}

Let $y = \sigma(x)$, we have
\begin{align}
& y = \frac{1}{1+\exp(-x)} \notag\\
\iff & \exp(-x) = \frac{1}{y} - 1 \notag\\
\iff & x = -\log\frac{1-y}{y} = \log\frac{y}{1-y} \notag\\
\iff & \sigma^{-1}(x) = \log\frac{x}{1-x}, \quad\forall x\in(0,1). \tag{3.38}
\end{align}

Let $y = \zeta(x)$, we have
\begin{align}
& y = \log(1+\exp(x)) \notag\\
\iff & \exp(x) = \exp(y) - 1 \notag\\
\iff & x = \log(\exp(y)-1) \notag\\
\iff & \zeta^{-1}(x) = \log(\exp(x)-1), \quad \forall x>0. \tag{3.39}
\end{align}

Also, we have
\begin{align}
\int^x_{-\infty}\sigma(y)dy &= \int^x_{-\infty}\frac{1}{1+\exp(-y)}dy \notag\\
&= \int^x_{-\infty}\frac{e^y}{e^y+1}dy \notag \\
& \xlongequal{z\triangleq e^{y}} \int^{e^x}_{0}\frac{1}{z+1}dz \notag\\
&= \log(z+1) | ^{e^x}_0 \notag\\
&= \log(\exp(x) + 1) \notag\\
&= \zeta(x). \tag{3.40}
\end{align}

Finally, we also have
\begin{align}
\zeta(x) - \zeta(-x) &= \log(1+e^x) - \log(1+e^{-x}) \notag\\
&= \log\frac{1+e^x}{1+e^{-x}} \notag\\
&= \log\frac{e^x(e^{-x}+1)}{1+e^{-x}} \notag\\
&= \log e^x \notag\\
&= x. \tag{3.41}
\end{align}

%-------------------------------------
%=====================
\end{document}
